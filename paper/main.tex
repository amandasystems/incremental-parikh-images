\documentclass[runningheads]{llncs}

\usepackage{amssymb,amsmath,mathtools,empheq,fancybox}

\usepackage{paralist}
\usepackage{url}
\usepackage{color}

\usepackage{textcomp,listings}

\newif\ifcomments
\commentstrue

\newcommand{\pr}[1]{{\color{red}``PR: #1''}}
\newcommand{\as}[1]{{\color{blue}``AS: #1''}}

\newif\ifoutline
\outlinetrue

\newcommand{\contents}[1]{\ifoutline{\color{blue}
    \begin{itemize}
    #1
    \end{itemize}
  }\fi}

\allowdisplaybreaks[1]

\title{Generalised Parikh Images of Regular Languages}
\author{some cool authors}
\institute{Uppsala University, Sweden}

\begin{document}
\maketitle

\begin{abstract}
  bla
\end{abstract}

\section{Introduction}

\contents{
\item key challenges: (1) complicated QE problem that was part of
  previous definitions; (2) languages with large alphabets described
  by symbolic automata; (3) regular languages described by products
  of automata
}

\subsection{Related Work}

\section{Overview}

\subsection{Generalised Parikh Images}

\contents{
\item Preliminaries, standard definition of the Parikh image of a
  regular/context-free language
\item Generalised version, considering arbitrary homomorphisms
\item Examples
}

\subsection{Lazy Computation of Parikh Images for Regular Languages}

\contents{
\item Lazy expansion of the Parikh conditions for a symbolic automaton
\item Finding elements vs.\ computing the complete Parikh image
\item Lazy product computation
}

\section{A Calculus of Parikh Images}

\contents{
\item Preliminaries, the underlying calculus, what are rules
\item Predicates used to represent Parikh images
\item Our calculus rules
\item Statement of properties, correctness, complexity
}

\section{Parikh Images from Products of Automata}

\contents{
\item additional rules needed for products
\item backjumping and conflict-driven learning
}

\section{Implementation and Experiments}

\contents{
\item length constraints
\item Parikh automata, automata with registers
\item Model-checking examples
}

\section{Conclusions}

\clearpage
%\bibliographystyle{splncs03}
%\bibliography{refs}

\end{document}
