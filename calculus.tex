We start by defining our calculus, \Calculus{}, for one automaton, and only
extend it to products of automata in \cref{sec:multiple}.
%
Assume an automaton $\Automaton = \AutomatonTuple$ with $\NrTransitions$
transitions $\Transitions =
\Set{\EllipsisSequence{\Transition}{\NrTransitions}}$. We use
the notations introduced in \cref{sec:languages}. For convenience, we
introduce the following additional notions:

\begin{definition}
  The \textit{transition count},
  $\TransitionCount(\Transition, \Path)$ is the number of times a
  transition
  $\Transition = \FromLabelTo{\State}{\Label}{\State'} \in
  \Transitions$ appears on a path $p$. A \emph{transition selection
    function} is a function~$\Filter:\: \Transitions \to \Naturals$
  labelling every transition~$\Transition \in \Transitions$ with a
  non-negative number.
\end{definition}

We introduce the two predicates that will be used by our calculus,
with the following definitions:
%
\begin{definition}\label{def:single-image}
  The Parikh predicate, $\SinglePredicateInstance$ holds for some
  automaton $\Automaton = \AutomatonTuple$, some map
  $\Map : \Alphabet^* \to \Monoid$ to a commutative monoid $\Monoid$,
  some transition selection function
  $\Filter:\: \Transitions \to \Naturals$, and some monoid
  element~$\MonoidElement \in \Monoid$ if $\MonoidElement$ is an
  element of the Parikh image of $\Language(\Automaton)$ modulo
  $\Map$, or more formally, when there is an accepting path
  $\Path = \PathEnumeration \in \Accepting{\Paths(\Automaton)}$ such
  that $\Filter(\Transition) = \TransitionCount(\Transition, \Path)$
  for all $t \in \Transitions$, and
  $\MonoidElement = \Map(\WordOf(\Path))$.
\end{definition}

\begin{definition}\label{def:connected} $\Connected(\Automaton, \Filter)$
  holds for some automaton $\Automaton = \AutomatonTuple$ and
  transition selection function
  $\Filter:\: \Transitions \to \Naturals$
  if for every
  $\Transition = \FromLabelTo{\State}{}{\State'} \in \Transitions$,
  if $\Filter(\Transition) > 0$ there is some $\Filter$-selected
  accepting path that visits $\Transition$'s starting state~$\State$. More formally,
  if $\Filter(\Transition) > 0$ then there is some
  path~$\Path \in \Paths(\Automaton)$ with
  $\Filter(\Transition') > 0$ for every $\Transition' \in p$, and $\State \in
  \StatesOf(p)$ such that $p$ ends in an accepting state $\State' \in \AcceptingStates$. The predicate~$\Connected(\Automaton, \Filter)$ represents the
  condition that $\Automaton$ is connected under the selection
  function $\Filter$ for every transition, and is implied by
  $\SinglePredicateInstance$.
\end{definition}


\begin{table}
  \caption{Derivation rules for one automaton.}\label{tbl:rules:single}
\begin{tabular}{@{}l>{$}c<{$}p{3cm}@{}}\toprule
  Name & \text{Rule} & Side conditions\\
  \midrule

  % EXPAND
  \Expand & 
    \inferrule
  {\Connected(\Automaton, \Filter) ,~ \FlowEq(\Automaton, \Filter) ,~ \MonoidElement = \sum_{\Transition \in \Transitions_\Automaton} \Filter(\Transition) \cdot \Map(\Transition), \SomeInequalities, \SomeClause}
  {\SinglePredicateInstance, \SomeInequalities, \SomeClause} & 
  None \\[4ex]

  % SPLIT
  \Split & 
  \inferrule{\Connected(\Automaton, \Filter), \SomeInequalities, \SomeClause, \Filter(\Transition) = 0 \mid \Connected(\Automaton, \Filter), \SomeInequalities, \SomeClause, \Filter(\Transition) > 0}{\Connected(\Automaton, \Filter), \SomeInequalities, \SomeClause} &
  if $\Transition \in \Transitions_\Automaton$ \\[4ex]

  % PROPAGATE
  \Propagate & \inferrule{\Connected(\Automaton, \Filter),
  \Set{\Filter(\Transition') = 0 \SuchThat \Transition' \in C},
  \SomeInequalities, \SomeClause, \Filter(\Transition) =
  0}{\Connected(\Automaton, \Filter), \Set{\Filter(\Transition') = 0 \SuchThat
  \Transition' \in C}, \SomeInequalities, \SomeClause} & if
                                                         $t \in \Transitions_\Automaton$ and\newline $\Dominates(C,
  \Automaton, \Transition)$ \\[4ex]

  % SUBSUME
  \Subsume &
  \inferrule{\SomeInequalities,\SomeClause}{\Connected(\Automaton, \Filter), \SomeInequalities, \SomeClause} &
  \Split{} and \Propagate{}\newline cannot be applied \\
  \bottomrule
  \end{tabular}
\end{table}

The rules of \Calculus{} for one automaton are given in
\cref{tbl:rules:single}. The rules operate on sets of formulas and can
be interpreted as rules of a one-sided sequent calculus, in which all
formulas are located in the antecedent~\cite{Fitting96a}.
%
The rules relate premises~$\SomeClause_1, \ldots, \SomeClause_k$ with
some conclusion~$\SomeClause$. 
%
When constructing a proof, we start with some root~$\SomeClause$, and
then apply proof rules to the goals of the proof, in bottom-up
direction, until all goals are closed, or no more rule is applicable.

A proof in which all proof goals are closed shows that the formulas
in the root~$\SomeClause$ of the proof are inconsistent, and do not
have any solutions. A proof goal that cannot be closed, but to which no
more rules are applicable, gives rise to  a solution of the formulas in the
root~$\SomeClause$. Such an open proof goal will only contain formulas
in Presburger arithmetic allowing a solution to be computed using
standard algorithms~\cite{DBLP:books/daglib/0022394}.

We use the convention of splitting the formulas in proof goals into
linear (in-) equalities ($\SomeInequalities$) and other formulas
($\SomeClause$), and assume that predicates $\Image$ and $\Connected$
only occur positively.
%Since our rules operate by adding and matching
%linear (in)equalities in a proof goal; we use the shorthand of listing
%the matched inequalities as antecedents, e.g., in \Propagate{}.
%
The transition selection function~$\Filter$ is represented
symbolically and can, in practice, be read as a function from
transitions to $\Naturals$-valued terms (e.g.~\texttt{t}
or~\texttt{t+1}). In our implementation \Catra{}, described 
in~\cref{sec:implementation}, $\Filter$~is a vector of fresh variables
with the same size as~$\Transitions$.

To ensure termination, rules can only be applied when they add new formulas
on every created branch (the notion of \emph{regularity} of a proof is
required~\cite{Fitting96a}). For example, this means that \Split{} can
only be applied to proof goals that contain neither
$\Filter(\Transition) = 0$ nor $\Filter(\Transition) > 0$, and can
never be applied to split on the same term twice on the same branch.

The rule~\textsc{Expand} expands an $\SinglePredicateInstance$
predicate into the more basic predicate
$\Connected(\Automaton, \Filter)$, as well as linear equations
relating the transitions mentioned by~$\Filter$ with the monoid
element~$\MonoidElement$, and linear flow equations described
by~\FlowEq{} (below). Since $\Connected$ and $\Image$ are partially
redundant and the difference is covered by~$\FlowEq$, we can remove
the instance of~$\Image$ when applying~$\Expand$. In this sense, we
split the semantics of the $\Image$~predicate into its counting aspect
(covered by $\FlowEq$) and its connectedness aspect (covered by
$\Connected$).

In \textsc{Expand}, we use the shorthand
notation~$\Map(\FromLabelTo{\State}{\Label}{\State'}) = \Map(\Label)$,
i.e., we allow the homomorphism~$h$ to be applied also to
transitions~$\Transition$.
%
The predicate $\FlowEq(\Automaton, \Filter)$ represents the flow
equations to be generated when expanding
$\SinglePredicateInstance$. We assume that each application of the
predicate introduces fresh integer variables~$f_q$ for every accepting
state~$q \in \AcceptingStates$, and define:
%
\[
\begin{aligned}
  \FlowEq(\Automaton, \Filter) &~=~
   \AndComp{\State \in \AcceptingStates}{\FinalStateVar_\State \geq 0}
   \land
   \sum\limits_{\State \in \AcceptingStates} \FinalStateVar_\State = 1 \land \AndComp{\State \in \States}{\In(\State, \Filter) - \Out(\State, \Filter)} = \Sink(\State)\\
  \Sink(\State) & ~=~ 0 \text{ if } \State \not\in \AcceptingStates, \FinalStateVar_\State \text{ otherwise.} \\
  \In(\State, \Filter) & ~=~ \StartFlow(\State) + \sum_{\Transition \in\, (\FromLabelTo{}{}{\State})} \Filter(\Transition)\\
  \StartFlow(\State)  &~=~ 1 \text{ if } \State = \InitialState, \text{ otherwise $0$.} \\
  \Out(\State, \Filter) &~=~ \sum_{\Transition \in\, (\FromLabelTo{\State\,}{}{})} \Filter(\Transition)
\end{aligned}
\]

The rule~$\Split{}$ allows us to branch the proof tree by trying to exclude a
transition from a potential solution before concluding that it must be included.
Intuitively, this is what guarantees our ability to make forward progress by
eliminating paths through~$\Automaton$.

The $\Propagate{}$ rule allows us to propagate (dis-)connectedness
across $\Automaton$. It states that we are only allowed to use
transitions attached to a reachable state, and is necessary to ensure
connectedness in the presence of cycles in~$\Automaton$. The rule
makes use of the notion of \emph{dominating} sets of transitions, defined
as follows:
%
\begin{definition}
  A set~$C$ of transitions of an automaton~$\Automaton$ \emph{dominates}
  a  transition~$\Transition$, written
  $\Dominates(C, \Automaton, \Transition)$, if
  every accepting path $p$ of $\Automaton$
  with $\Transition \in p$ contains at least one transition from
  $C$. Notably, $\Dominates(\emptyset, \Automaton, \Transition)$ for
  every unreachable transition $\Transition$, and
  $\Dominates(\Set{\Transition}, \Automaton, \Transition)$ for every
  transition~$\Transition$.
\end{definition}

The $\Dominates$~relation can be efficiently implemented in a solver
by using standard Ford-Fulkerson\slash{}Edmonds-Karp min-cut between a state
and the initial state after removing transitions where $\Filter(\Transition) = 0$.
By only performing this computation after such filtering, a solver additionally avoids
breaking the rule of adding clauses that already appear in the formula.

We here only define forward propagation. Backward propagation 
can be straightforwardly obtained by computing reachability from
an accepting state on the same principle as above, and indeed our
implementation of \Calculus{} does this.

Finally, the rule~\Subsume{} can be applied when the connectedness of
an automaton has been ensured by exhaustive application of the other
rules. This suggests a proof strategy where you \Propagate{} when you
can, \Split{} when you must, and \Subsume{} when neither is possible
anymore.

In addition to \cref{tbl:rules:single}, we assume the existence of a rule
\PresburgerClose{}, corresponding to a sound and complete solver for Presburger
formulae, and for constraints over~$\Monoid$.

A decision procedure would start from one or multiple
predicates~$\SinglePredicateInstance$ to be satisfied, possibly in
combination with other constraints about $m$. It would then first
expand the predicates using the $\Expand{}$~rule, and subsequently
apply the other rules to search for a solution.

As illustrated in \cref{sec:intuition}, a decision procedure can
also perform arithmetic rewriting of the occurring terms and
equations. Such reasoning is not necessary for correctness or
completeness, but it shortens the examples considerably; we
will therefore assume the existence of a
rule \EquationReasoning{} that allows us to perform standard algebraic reasoning
on linear arithmetic constraints. 

%, boiling them down to choices of branches, which depend
%on one single variable and loop transitions. This logic corresponds to
%the placement of counters for optimally edge-profiling the CFG of a
%program, making up a minimum-spanning tree of the
%automaton~\cite{path-profiling}.

\subsection{An Example}\label{sec:single:example}
Here we will return to $\mathcal{A}'$ from \cref{sec:intuition} and perform the steps of
\cref{sec:intuition:algebra,sec:intuition:split} with the
formal calculus we just established, but exclude \cref{const:s2-in-b} and
therefore the entire automaton $\mathcal{B}$, since we introduce support for products of
automata in \cref{sec:multiple}. The example then becomes
satisfiable, as $s_1 = \Char{dd}, i = 1, s_2 = \Char{adda}, l_A = 4$ is a
satisfying assignment to
\cref{const:more-inside-than-before,const:s1-in-c-dd,const:s1-substring,const:something-before-and-after}.

Starting with~$\mathcal{A}'$, $\Map$ extracts the increments of a transition, $h(\FromLabelTo{q_1}{c / i}{q_2}) = \Vec{v}$ for the increment
vector $\Vec{v}$, concretely $h(\FromLabelTo{I_A}{\Sigma / \left[ l_{A+}, i_+
\right]}{I_A}) = \left[ l_{A+}, i_+ \right]$. We use the same compact notation
here as in \cref{sec:intuition} to represent what is essentially a sparse vector
of $1$ and $0$ coefficients, i.e. $\left[ l_{A+}, i_+ \right] = \left[1, 1,
0\right]$.  The reader is
advised to review \cref{fig:aut_a} from
\cref{sec:intuition} while going through this example.

Initially, we let $\Filter$ map to fresh variables to obtain, after some simplifications:
\begin{equation}\label{eq:single-filter}
  \begin{array}{rl@{\qquad}rl@{\qquad}rl}
    \Filter(\FromLabelTo{D}{d/[l_{A+}, n_+]}{F_A})
    &= \TransitionVar_1
    &
    \Filter(\FromLabelTo{F_A}{\Sigma/[ l_{A+} ]}{F_A} )
    &= \TransitionVar_2   
    &
    \Filter(\FromLabelTo{I_A}{c/[l_{A+}, n_+]}{F_A} ) 
    &=  \TransitionVar_3 
    \\
    \Filter(\FromLabelTo{I_A}{d/[l_{A+}, n_+]}{D}) 
    &= \TransitionVar_4
    &
    \Filter(\FromLabelTo{I_A}{\Sigma/[ l_{A+}, i_+]}{I_A}) 
    &= \TransitionVar_5  \\[3ex]
%
  \multicolumn{6}{c}{
    \FlowEq(\mathcal{A}', \Filter)  =\quad \begin{aligned}
                       1 + \cancel{\TransitionVar_5} &= \cancel{\TransitionVar_5} + \TransitionVar_4 + \TransitionVar_3 \wedge\mbox{}\\
                    \TransitionVar_4 &= \TransitionVar_1 \wedge\mbox{}\\
                    \TransitionVar_3 + \TransitionVar_1 + \cancel{\TransitionVar_2} &= \cancel{\TransitionVar_2} + 1  \\
                    \end{aligned}}
  \end{array}
\end{equation}


\begin{figure}[h]
  \centering
\begin{prooftree}
  \hypo[]{$\vdots$}
  \infer1[]{
      x_1 = 1 \land
      x_2 = l_A - 3 \land
      x_3 = 0 \land 
      x_4 = 1 \land
      x_5 = 1 \land
      n = 2 \land i = 1
    \land l_A > 3
  }
  \infer1[\Subsume]{
    \begin{array}{c}
      \Connected(A', \Filter) \land \mbox{}\\
      x_1 = 1 \land
      x_2 = l_A - 3 \land
      x_3 = 0 \land 
      x_4 = 1 \land
      x_5 = 1 \land
      n = 2 \land i = 1
    \land l_A > 3
    \end{array}
%    \begin{aligned}
%    i = \TransitionVar_5 \land 
%    n = 2 \land \\
%    0 < i < 2
%    \land l_A - i - 2 > 0 
%    \land \TransitionVar_2 = l_A - i - 2 
%    \land \TransitionVar_3 = 0 \\
%  \end{aligned}
  }
  \infer1[\EquationReasoning: $=$-simp.]{
    \begin{array}{c}
      \Connected(A', \Filter) \land\mbox{}\\
      x_1 = n - 1 \land
      x_2 = l_A - n - i \land
      x_3 = 2 - n \land 
      x_4 = n - 1 \land
      x_5 = i \land \mbox{}\\
      n = 2 \land i = 1
    \land l_A > 3
    \end{array}
%    \begin{aligned}
%    \Connected(\mathcal{A}', \Filter) \land 
%    \TransitionVar_1 = 1 \land 
%    i = \TransitionVar_5 \land 
%    n = 2 \land \\
%    0 < i < 2 
%    \land l_A - i - 2 > 0
%    \land \TransitionVar_2 = l_A - i - 2 
%    \land \TransitionVar_3 = 0
%    \end{aligned}
  }
  \infer1[\EquationReasoning{}: $>$-reasoning]{
    \begin{array}{c}
      \Connected(A', \Filter) \land\mbox{}\\
      x_1 = n - 1 \land
      x_2 = l_A - n - i \land
      x_3 = 2 - n \land 
      x_4 = n - 1 \land
      x_5 = i \land \mbox{}\\
      n > i 
    \land l_A - i - n > 0
    \land i > 0
    \end{array}
%    \begin{aligned}
%    \Connected(\mathcal{A}', \Filter) \land
%    i = \TransitionVar_5 \land 
%    n = 2\TransitionVar_1 
%    \land n > i > 0 \\
%    \land l_A - i - n > 0 
%    \land \TransitionVar_2 = l_A - i - n 
%    \land \TransitionVar_3 = 1 - \TransitionVar_1 
%    \end{aligned}
  }
  \infer1[\EquationReasoning{}: $=$-simp.]{
    \begin{array}{c}
      \Connected(A', \Filter) \land\mbox{}\\
      1 = x_4 + x_3 \land x_4 = x_1 \land x_3 + x_1 = 1 \land \mbox{}\\
      l_A = x_1 + x_2 + x_3 + x_4 + x_5 \land
      i = x_5 \land 
      n = x_1 + x_3 + x_4 \land \mbox{}\\
      n > i 
    \land l_A - i - n > 0
    \land i > 0
    \end{array}
%    \begin{aligned}
%    \Connected(\mathcal{A}', \Filter)  
%    \land i = \TransitionVar_5 
%    \land n = \TransitionVar_4 + \TransitionVar_1 
%    \land n > i > 0
%    \land l_A - i - n > 0
%    \land \TransitionVar_4 = \TransitionVar_1 \\
%    \land \TransitionVar_2 = l_A - i - n 
%    \land \TransitionVar_3 = 1 - \TransitionVar_1
%    \land 1 = \TransitionVar_3 + \TransitionVar_1 
%    \land i > 0
%    \land \TransitionVar_4 = \TransitionVar_1 
%    \end{aligned}
  }
  \infer1[(Expanding $\FlowEq$)]{
    \begin{array}{c}
      \Connected(A', \Filter) \land
      \FlowEq(A', \Filter) \land \mbox{}\\
      l_A = x_1 + x_2 + x_3 + x_4 + x_5 \land
      i = x_5 \land 
      n = x_1 + x_3 + x_4 \land \mbox{}\\
      n > i 
    \land l_A - i - n > 0
    \land i > 0
    \end{array}
%    \begin{aligned}
%    i = \TransitionVar_5 
%    \land l_A = \sum_{k = 1}^{k = 5} \TransitionVar_k 
%    \land n > i > 0
%    \land l_A - i - n > 0
%    \land n = \TransitionVar_4 + \TransitionVar_1\\
%    \land \Connected(\mathcal{A}', \Filter) 
%    \land \FlowEq(\mathcal{A}', \Filter)
%    \end{aligned}
  }
  \infer1[(Expanding sum)]{
    \begin{array}{c}
      \Connected(A', \Filter) \land
      \FlowEq(A', \Filter) \land
      \left[l_A, i, n\right] =
      \sum\limits_{\Transition \in \Transitions_{\mathcal{A}'} } \Map(\Transition) \cdot \Filter(\Transition) \land\mbox{}\\
     n > i 
    \land l_A - i - n > 0
    \land i > 0
    \end{array}
  }
  \infer1[\Expand{}]{
    \Image{}_{\mathcal{A}', \Map}(\Filter, \left[l_A, i, n\right]) 
    \land n > i 
    \land l_A - i - n > 0
    \land i > 0
    }
\end{prooftree}
\caption{A proof tree for
\cref{const:s1-in-c-dd,const:s1-substring,const:something-before-and-after,const:more-inside-than-before}
from \cref{ex:string-constraints}, corresponding to handling the Parikh
automaton $\mathcal{A}'$ of \cref{fig:aut_a}.}\label{fig:derivation:single}
\Description[]{}% This is fine; it's all text!
\end{figure}

The proof tree is shown in~\cref{fig:derivation:single}. Like in
\cref{sec:a_1,sec:intuition:algebra}, by arithmetic reasoning the
calculus ends up with fixed values for several of the transition
variables $\TransitionVar_k$, and eventually concludes that the
transition directly from state $I_{A}$ to $F_{A}$ (variable~$x_3$)
is incompatible with
the constraint.  It is then possible to remove the $\Connected{}$
predicate using \Subsume{}, since \Propagate{} is not able to infer
further constraints, and no non-trivial applications of \Split{}
remain to be done.  After this, by further arithmetic reasoning we can
derive a solution~$l_A = 4, i = 1, n = 2$, and conclude that the root
constraint is satisfiable.  To obtain values for the string
variables~$s_1, s_2$ corresponding to the solution, one can
construct an accepting path of the automaton with each transition~$t$
taken $\Filter(t)$ times.

\iffalse

the rest of the reasoning can be
continued without any part of \Calculus{} since all values of the
existentially quantified transition variables
$\TransitionVar_1,\ldots,\TransitionVar_5$ compatible with
\cref{const:more-inside-than-before,const:s1-in-c-dd,const:s1-substring,const:something-before-and-after}
are represented by the flow equations introduced by $\Expand{}$.
\fi

\subsection{Correctness of \Calculus{}}\label{sec:single:correct}

Our correctness proof of \Calculus{} consists of two main parts: first, we show
that the construction of a proof always terminates, and then that each of the
proof rules in \cref{tbl:rules:single} is an equivalence transformation, i.e.,
does not change the set of satisfying assignments of a formula. In combination,
those two results immediately imply that \Calculus{} gives rise to a decision
procedure.

\subsubsection{\Calculus{} terminates}
\begin{lemma}\label{lma:single-terminates}
  Suppose $\SomeClause{}$ is a set of formulas in which the predicates
  $\Image$ and $\Connected$ only occur positively. There is no
  infinite sequence of proofs~$P_0, P_1, P_2, \ldots$ in which $P_0$
  has $\SomeClause{}$ as root, and each $P_{i+1}$ is derived from
  $P_i$ by applying one of the rules in \cref{tbl:rules:single}.
\end{lemma}

\begin{proof}
  The rule~\Expand{} can only be applied finitely often since each
  application removes one $\Image$ predicate, and none of the rules
  introduce new instances of the predicate. The rule~\Subsume{} can
  only be applied finitely often since it strictly decreases the
  combined number of $\Image$ and $\Connected$ predicates in sets of
  formulas, and none of the rules increases that number.

  To show termination of \Split{} and \Propagate{}, observe that the
  $\Filter$ in a predicate~$\Connected(\Automaton, \Filter)$ is never
  updated on a proof branch, which means that the set of terms
  $\Filter(t)$ for $t \in \Automaton$ on every branch is finite. Each
  application of \Split{} and \Propagate{} adds a new
  formula~$\Filter(\Transition) = 0$ or $\Filter(\Transition) > 0$
  to a proof goal, which can only happen finitely often.
\end{proof}

\subsubsection{The rules in \cref{tbl:rules:single} are solution-preserving}

\begin{lemma}\label{lma:single-correct}
  Consider an application of one of the rules in
  \cref{tbl:rules:single}, with
  premises~$\SomeClause_1, \ldots, \SomeClause_k$ and
  conclusion~$\SomeClause$. An assignment~$\beta$ satisfies the
  conclusion~$\SomeClause$ if and only if it satisfies one of the
  premises~$\SomeClause_i$.
\end{lemma}

\begin{proof}
  This property has to be shown by analysing the possible applications
  of each proof rule.

  \Expand{} unfolds the definition of the $\Image$ predicate. To show
  that the rule is solution-preserving, we prove the equivalence of the
  upper and lower sets of formulas:
  \begin{itemize}
  \item Assume that $\beta$ satisfies the conclusion, which means that
    there is some accepting path
    $\Path = \PathEnumeration \in \Accepting{\Paths(\Automaton)}$ with
    $\val_\beta(\Filter(\Transition)) = \TransitionCount(\Transition,
    \Path)$ and $\val_\beta(\MonoidElement) =
    \Map(\WordOf(\Path))$. This immediately implies that $\beta$
    satisfies $\Connected(\Automaton, \Filter)$, since a path is
    connected, and $\FlowEq(\Automaton, \Filter)$ since an accepting
    path satisfies the flow equations. The
    equation~$\MonoidElement = \sum_{\Transition \in
      \Transitions_\Automaton}\Filter(\Transition) \cdot
    \Map(\Transition)$ holds because of
    $\val_\beta(\Filter(\Transition)) = \TransitionCount(\Transition,
    \Path)$.
  \item Assume that $\beta$ satisfies the premise, which implies that
    $\val_\beta(\Filter)$ describes a consistent, connected flow of
    the automaton. By the same argument as in
    \cite{generate-parikh-image}, this flow
    can be mapped to an accepting path~$\Path$ of $\Automaton$ such
    that each transition~$\Transition$ occurs on $\Path$ exactly
    $\val_\beta(\Filter(\Transition))$ times. Together with the equation
    $\val_\beta(\Filter(\Transition)) = \TransitionCount(\Transition,
    \Path)$, this implies that $\beta$ satisfies $\SinglePredicateInstance$.
  \end{itemize}

  In \Split{}, we make use of the fact that $\Filter(\Transition)$ is
  $\Naturals$-valued by definition. For any $\beta$, clearly exactly
  one of $\Filter(\Transition) = 0$ or $\Filter(\Transition) > 0$ will
  be satisfied, implying the property.

  For \Propagate{}, suppose that $\Dominates(C, \Automaton, \Transition)$, which
  means that every accepting path containing~$\Transition$ contains at least one
  of the transitions in $C$. For a $\beta$ satisfying $\Connected(\Automaton,
  \Filter)$. This means that every accepting path $p$ where $\Transition \in p$
  has at least one transition $\Transition' \in C$ such that
  $\Filter(\Transition') = 0$. This means that also
  $\val_\beta(\Filter(\Transition)) = 0$ has to hold since no unbroken path
  containing $\Transition$ exists.

  Finally, for \Subsume{}, observe that if \Split{} cannot be applied,
  then a goal must contain~$\Filter(\Transition) = 0$ or
  $\Filter(\Transition) > 0$ for every $\Transition$. In case the
  formulas in $\SomeInequalities$ are inconsistent, an application of
  \Subsume{} is trivially solution-preserving; therefore assume that
  $\SomeInequalities$ is consistent, which means that it contains
  exactly one of $\Filter(\Transition) = 0$ or
  $\Filter(\Transition) > 0$ for each $\Transition$. Since
  \Propagate{} is not applicable, the transitions~$\Transition$ with
  $\Filter(\Transition) > 0$ must form a connect sub-graph of the
  automaton; this means that $\Connected(\Automaton, \Filter)$ is
  redundant as it is implied by $\SomeInequalities$.
\end{proof}

%%% Local Variables:
%%% mode: latex
%%% TeX-master: "main"
%%% End:
