The Parikh image is a characterisation of formal languages in terms of their
character counts. Given a language over an alphabet~$\{a_1, \ldots, a_k\}$, the
Parikh image is a set of $k$-dimensional vectors that contains some
vector~$\VectorLiteral{m_1, m_2, \ldots, m_k}$ if and only if the formal
language contains a word in which each $a_i$ occurs $m_i$ times. It is a
classical result that the Parikh image of every context-free language (and,
thus, also of every regular language) is a semilinear set~\cite{parikh-theorem},
i.e., Presburger-definable.

Parikh images play a central role in many automata-based algorithms, for
instance and notably in today's string solvers, which often have to process
constraints that combine regular language membership with word length. To decide
whether a simple formula like $x \in \Language_1 \wedge y \in \Language_2 \wedge
|x| > |y|$, with string variables~$x, y$ and regular languages~$\Language_1,
\Language_2$, is satisfiable, it is necessary to reason about the sets of word
lengths induced by $\Language_1, \Language_2$, which are special cases of the
Parikh image.  This required combined reasoning about strings and string length
has long been identified as a major bottleneck in string solvers
\cite{DBLP:conf/cav/AbdullaACHRRS15,length-aware-solver,approximate-parikh,DBLP:journals/corr/BerzishZG17}.
Other string solvers make use of Parikh automata~\cite{parikh-automata}, and
thus Parikh images in the general case, to handle operations including
\verb!str.substr!  and \verb!str.at!, which comes at an even higher price in
terms of computational complexity~\cite{ostrich-plus}.


It is possible to compute an existential Presburger formula describing the
Parikh image of any context-free language in linear
time~\cite{generate-parikh-image}; for the special case of regular languages,
this result was also stated in \cite{muscholl-linear}. While theoretically
elegant, this construction has several disadvantages, often making it
unpractical for integration into algorithms. Firstly, the constructed Presburger
formula contains a linear number of existential quantifiers in the size of the
considered grammar, as well as complex Boolean structure, which is needed to
express the connectedness of sets of productions considered in the construction.
Eliminating those quantifiers to obtain a quantifier-free representation of the
Parikh image has exponential complexity~\cite{DBLP:conf/issac/Weispfenning97},
and is often impossible in reasonable time. Just solving the Parikh image
membership problem is NP-complete, as it corresponds to computing a satisfying
assignment of the existential Presburger formula, and taxing for solvers as
well~\cite{ostrich-plus}.

Secondly, in applications involving regular languages, it is typically necessary
to consider the Parikh image not only of a single automaton, but of the
intersection of multiple automata. This problem arises in string solvers in
particular, as conjunctions of string constraints lead to the computation of
length images of products (intersections) of regular languages represented as
finite automata. Applying the approach in~\cite{generate-parikh-image} would in
this case require the eager computation of the product before its length image,
and result in an existential Presburger formula of exponential size (in the
number of automata). In several instances we have observed while solving
real-world string constraints, the computation of the product of automata
exhausts the memory of any machine available due to the exponential blow-up in
size of the product, quickly becoming intractable as the number of automata in
the product increases. The current best published mitigation for this problem is
an over-approximation that works by approximating the Parikh image of a product
of automata to be the conjunction of the image of the individual automata of the
product \cite{approximate-parikh}. This approach only works for unsatisfiable
instances, and comes with a harsh penalty for satisfiable instances.

Addressing these concerns, we have developed a calculus for Parikh images of
products of regular languages that we call \Calculus{}. It allows us to
interleave the computation of arbitrarily deep products of automata with the
product's Parikh image, and is generalised to an arbitrary homomorphism over
automata labels, including string lengths. This enables us to let both
calculations inform each other, eliminating unnecessary work, and pruning the
size of the partial products considered in the computation for a smaller memory
footprint. Moreover, the method can be used iteratively to tackle smaller chunks
of the product incrementally, thereby decreasing the memory footprint.

The key ideas of \Calculus{} is a combination of problem-aware case splitting,
lazy enforcement of automata connectivity, and lazy computation of products. In
essence, we use constraint-based methods to efficiently enforce the constraints
associated with computing Parikh images of regular languages.

We implement \Calculus{} as a plug-in theory for the \Princess{} automated
theorem prover. \Fudge{Others} have integrated it into the similarly
\Princess{}-backed string constraint solver \OstrichPlus{}, where it is used to
handle constraints that use cost-enriched automata, as described in
\cite{ostrich-plus}.

We additionally wrap the Parikh image solver as a stand-alone tool, \Catra.
\Catra{} also supports a variant of the approximate method of
\cite{approximate-parikh}, its fall-back variant adapted
from~\cite{generate-parikh-image}, and an adapter for the \Nuxmv{} model
checker~\cite{nuxmv}. Using \Catra, we compare \Calculus{} to the other two
back-ends on \NrBenchmarks{} distinct Parikh automata intersection problems
generated by \OstrichPlus{} when solving the PyEx string constraint benchmark
suite involving string length constraints \cite{pyex}.

In summary, we contribute:
\begin{itemize}
    \item The \Calculus{} calculus to efficiently compute (a homomorphism on)
          the Parikh image of products of Parikh automata.
    \item Experiments illustrating the performance of \Calculus{} on real-world
    examples from string solving, including \NrBenchmarks{} instances in a
    standardised format made available for future study.
    \item The \Catra{} tool for solving such instances, containing an
    implementation of \Calculus{}, the over-approximation described
    in~\cite{approximate-parikh}, and an adapter for the~\Nuxmv{} model
    checker~\cite{nuxmv}.
    \item Suggestions for how to efficiently implement \Calculus{} in a modern
    automated theorem prover, including strategies for case splitting, clause
    learning, and constraint propagation for connectedness.
\end{itemize}

\subsection{Related Work}

The problem of computing constraints on Parikh images over products of regular
languages under a given commutative homomorphism amounts to solving products of
Parikh automata. Parikh automata are regular automata extended with integer
counters with given increments and decrements for each transition, where we
allow checking a set of linear constraints on the final values of the counters
(but not their intermittent values) \cite{parikh-automata}. Parikh automata
without constraints on the final values on their registers are also sometimes
called cost-enriched automata, weighted automata or counter automata, depending
on exact definitions and side-constraints. The decision problem tackled in this
paper, determining the emptiness of an intersection of Parikh automata, was
recently shown to be PSPACE-complete~\cite{graph-queries}.

Parikh image computations as well as Parikh automata feature extensively in
string solvers, including as mentioned above \Ostrich{} and \OstrichPlus{}
\cite{ostrich,ostrich-plus}, but also forms the basis of Trau~\cite{trau-pldi},
and occurs in \textsc{Sloth}~\cite{sloth}. Parikh images frequently appear when
introducing cardinality constraints like length or string indexing. The
state-of-the-art approach to handling Parikh image computation is to
over-approximate the Parikh image of a product of $k$~automata
$\ParikhMap(\Language(\Automaton_1) \cap \ldots \cap \Language(\Automaton_k))$
with the conjunction of the automata's parikh maps, $\bigwedge_{i=1}^{k}
\ParikhMap(\Automaton_i)$. This approach works only for unsatisfiable instances,
and will require falling back to computing the product of the automata before
using the standard approach for finding its image originally presented
in~\cite{generate-parikh-image}.

Outside of string solvers, Parikh automata have been proposed as the basis of
queries~\cite{graph-queries}, and for solving cardinalities in model checking
problems involving epistemic logic~\cite{epistemic-logic}.

Many other generalisations of the Parikh image than the projections we use here
have been studied. Prominent examples include generalising the Parikh map to
segments of a fixed length \cite{KARHUMAKI1980155} and the more general Parikh
matrix, which gives more information about a word than the standard Parikh image
\cite{parikh-matrix}. Another notable generalisation is the p-vector, introduced
in~\cite{infinite-words}, which denotes the position of each letter in the word
rather than the number of their occurrences and allows for generalisations into
infinite alphabets. All of these in some sense extend the Parikh map. By
contrast, the main utility of the formulation introduced here is that it allows
us to \emph{remove} something, thereby potentialy obtaining an easier problem.

\subsection{An Intuition for Our Approach}\label{sec:intuition}

\begin{figure}[h]
    \centering 
  \begin{subfigure}[b]{0.5\textwidth}
    \centering
    \includegraphics[scale=\autscale]{a}
    \caption{The automaton $A'$, whose Parikh registers count both the number of
    letters and the start ($l$) and length ($r$) of the substring matching
    $\Regex{c\RegexOr{}dd}$.}\label{fig:aut_a}
      %\Description[SHORT]{LONG}
  \end{subfigure}
  \begin{subfigure}[b]{0.5\textwidth}
    \centering
    \includegraphics[scale=\autscale]{b}
    \caption{The automaton $B$, where the Parikh registers count the number of
    occurrences of each letter.}\label{fig:aut_b}
      %\Description[SHORT]{LONG}
  \end{subfigure}
  \caption{The collection of automata we use as running
  examples}\label{fig:examples}
\end{figure}


In this section we will introduce an intuition for how a string constraint
problem in an automata-based solver like \OstrichPlus{} is translated to a
Parikh automata intersection problem and solved efficiently using \Calculus{}.
First, by a Parikh automaton we mean a standard non-deterministic finite
automaton (NFA) with a set of Parikh registers that are incremented (or
decremented) at each transition. In the automata of \cref{fig:examples}, we will
use the notation $a / \begin{bmatrix} x_+ \end{bmatrix}$ to mean that a
transition reads an input character $a$ and increments register $x$ by one. We
will omit any zero-valued increments and assume that all registers are scoped to
their automaton. The increments are usually represented as a vector (hence the
notation), but as it is mostly sparse here, we instead use the symbolic notation
$\begin{bmatrix} x_{1+} \end{bmatrix}$ rather than the more cumbersome $\left[
0, 1,  0 , \ldots, 0 \right]$.

Furthermore, we use $\CountOf{c}(A)$ to refer to the number of letters $c$ in
$\Language(A)$, and allow this expression to be used both in linear constraints
of the input language of \cref{ex:string-constraints} and in the langauge of
constraints that we translate to. Where the automaton $A$ is apparent, we will
omit it.

We use standard PCRE regular expressions here and throughout the paper, writing
them $\Regex{\texttt{in brackets}}$ to make that apparent. As is common,
$\RegexOr{}$ is alternation, $\mathtt{*}$ the Kleene star, and $\mathtt{.}$,
matches any single character.

\begin{example}\label{ex:string-constraints} Consider the following set of
    string constraints:
\begin{constraints}
    \item\label{const:x-in-c-dd} $x \in \Language(\Regex{c\RegexOr{}dd})$.
    \item\label{const:y-in-b} $y \in \Language(B)$, the language accepted by
    automaton~$B$ of \cref{fig:aut_b}.
    \item\label{const:x-substring} $x = \Substring(y, i, n)$, that is $x$ is an
    $n$-length substring of $y$ starting at offset $i$.
    \item\label{const:more-a-than-b} $i = n$, that is the substring is as long as its starting offset.
\end{constraints}
\end{example}

Intuitively, we see that the constraints are satisfiable.
\cref{const:more-a-than-b} implies that the starting offset must be either $1$
or $2$, that is we need to fill out $y$ with either one or two symbols before
the substring $x$ starts. The state $D$ is dead, as $B$ contains no transitions
with label $\Char{d}$, and hence none of the transitions into $D$ can appear in
the intersection of the languages. An example satisfying assignment is $y =
\Char{acca}$, which gives $x = \Char{c}$, $i = 1$, $n = 1$, but others are
possible.

To solve \cref{ex:string-constraints} using \Calculus{},
\cref{const:x-in-c-dd,const:y-in-b,const:x-substring} can be translated to the
product of Parikh automata $B \times A'$, where $A'$ (\cref{fig:aut_a}) is the
Parikh automata obtained by applying the encoding of $\Substring$ described in
\cite{ostrich-plus}. That description can then be used to obtain values for the
subsstring offsets $i, n$. Here follows an intuition for our method on $B \times
A'$ under the global constraint on the substring \cref{const:more-a-than-b}.

Our approach to the computation of the Parikh image (and therefore the solution
to the intersection problem posed by \cref{ex:string-constraints}) consists of
two applications of laziness; late computation of products of automata, and lazy
enforcement of connectivity constraints. Both are in contrast to the eager
approach to finding the Parikh image described in~\cite{generate-parikh-image},
that would have started by computing the product $A' \times B$, translated it to
a Presburger formula with $i, n$ etc as free variables, and then plugged in the
constraint \cref{const:more-a-than-b}.

Starting with an interactive theorem prover and the goal of solving the
constraint $\CountOf{\Char{a}}(B \times A') > \CountOf{\Char{b}}(B \times A')$
we associate each transition of $A'$ and $B$ with a fresh integer variable,
$\TransitionVar_\Transition$, identically to \cite{generate-parikh-image}. These
variables represent how many times each transition is taken. Hence, the final
counter value for each automaton is the element-wise sum:
\begin{equation}
\begin{bmatrix} 
  \CountOf{\Char{a}} \\ \CountOf{\Char{b}} 
  \\ \CountOf{\Char{c}} \\ \CountOf{\Char{d}} 
\end{bmatrix} = \sum_\Transition \TransitionVar_\Transition \cdot 
  \IncrementVec_\Transition \text{ where $\IncrementVec_\Transition$ are the increments of transition $\Transition$}.
\end{equation}

\begin{figure}[h]
    \centering 
  \begin{subfigure}[b]{0.5\textwidth}
    \centering
    \includegraphics[scale=\autscale]{a_annotated}
    \caption{ $A'$ with its associated transition variables in symbolic form.
    Note: unused ($0$) self-loops at $I_{A'}$ and $F_{A'}$ have been pruned for
    readability.}\label{fig:aut_a_annotated}
      %\Description[SHORT]{LONG}
  \end{subfigure}
  \begin{subfigure}[b]{0.5\textwidth}
    \centering
    \includegraphics[scale=\autscale]{b_annotated}
    \caption{$B$ with its associated transition variables in symbolic forms.
    Note that state $B$ is now completely dead.}\label{fig:aut_b_annotated}
      %\Description[SHORT]{LONG}
  \end{subfigure}
  \caption{The starting automata, without their counter increments and with
  their transition variables in symbolic form.}\label{fig:propagated}
\end{figure}


We proceed by adding linear constraints requiring transition variables to
represent a flow through either automata by requiring that the number of
incoming transitions is equal to the number of outgoing ones. E.g. state $B$
would have the sum $\cancel{x_{\Tuple{B, \Char{c}, B}}} + x_{\Tuple{I_B,
\Char{b}, B}} = \cancel{x_{\Tuple{B, \Char{c}, B}}}  + x_{\Tuple{B, \Char{c},
F_B}} + x_{\Tuple{B, \Char{b}, I_B}}$ where we let $x_{\Tuple{q, l, c'}}$ refer
to the integer variable associated with a transition from state $q$ to state
$q'$ with label $l$. Note that the self-loop cancels itself out. This means that
for automata where a loop can become unreachable, these constraints do not fully
capture the constraints posed by the automaton, but requires additional
reachability constraints.

Performing symbolic reasoning on these equations and plugging the corresponding
symbolic representation of the constraints on each corresponding integer
variable back as an annotation on its transition, we obtain the two automata in
\cref{fig:propagated}.

% FIXME B-loopen inte nollad här! Uppdatera!

Several transitions are already marked as unuseable (have a transition variable
value of $0$). This includes any transition for the letter $\Char{d}$, as
$\CountOf{\Char{d}}(B) = \sum_{\Transition} 0 = 0$, since the register for
$\Char{d}$ is zero for all transitions in $B$.

Note that the self-loop on state $B$ is nonzero due to the above mentioned
self-balancing. However, a reachability computation will conclude that the state
is unreachable, and allow us to conclude that all its outgoing transitions must
be zero.

After removing all transitions concluded to be unuseable to produce new,
equivalent automata, we compute the product in \cref{fig:product}.

\begin{figure}[h]
    \centering 
    \includegraphics[scale=\autscale]{ab}
  \caption{An automaton encoding the constraint $\CountOf{\Char{a}}(B \times A')
  > \CountOf{\Char{b}}(B \times A')$.}\label{fig:product}
      %\Description[SHORT]{LONG}
\end{figure}

By putting off computing the product $B \times A'$ until after performing linear
reasoning on the number of times each transition must be taken and using that to
prune the terms, we have computed a smaller product than we would have with a
na\"ive approach. Moreover, since this automaton contains only one loop along
one main path, all runs through it can be described by the flow-preservation
equations described above without the need for costly connectivity-enforcing
constraints as in~\cite{generate-parikh-image}. We can now obtain a generalised
solution to our constraints by inspection.