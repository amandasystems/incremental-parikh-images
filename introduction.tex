Extending automated theorem provers and SMT solvers with support for rich string
constraints is important for program analysis, particularly to detect cross-site
scripting vulnerabilities and other string manipulation
bugs~\cite{DBLP:books/sp/BultanYAA17}.
%
To check the satisfiability of a formula like $x \in \Language_1 \wedge y \in \Language_2
\wedge |x| > |y|$, with string variables~$x, y$ and regular
languages~$\Language_1, \Language_2$, it is necessary to reason
about the possible lengths of $x, y$ that are admitted by
$\Language_1, \Language_2$, respectively. The
set of word lengths $\{ |w| \mid w \in \Language\}$ in a language~$\Language$
is a special case of the Parikh image of a regular language.
This required combined reasoning about
strings and string length has long been identified as a major bottleneck in
string solvers
\cite{DBLP:conf/cav/AbdullaACHRRS15,length-aware-solver,approximate-parikh,DBLP:journals/corr/BerzishZG17}.
Other string solvers make use of Parikh automata~\cite{parikh-automata}, and
thus Parikh images in the general case, to handle operations that combine
strings and integers (including
\verb!str.substr!  and \verb!str.at!), which comes at an even higher price in
terms of computational complexity~\cite{ostrich-plus}.

The Parikh image, more broadly, is a characterisation of formal
languages in terms of their character counts. Given a
language~$\Language$ over an alphabet~$\{a_1, \ldots, a_k\}$, the
Parikh image is a set of $k$-dimensional vectors that contains some
vector~$\VectorLiteral{m_1, \ldots, m_k}$ if and only if the
language~$\Language$ contains a word in which each $a_i$ occurs $m_i$
times. It is a classical result that the Parikh image of every
context-free language (and, thus, also of every regular language) is a
semilinear set, i.e., Presburger-definable~\cite{parikh-theorem}.
In
fact, it is possible to compute an existential Presburger arithmetic formula
describing the Parikh image of any \emph{context-free} language in
linear time~\cite{generate-parikh-image} in the size of the grammar
describing the language. For the special case of
regular languages, this result was also stated
in~\cite{muscholl-linear}.

In applications,  it is often necessary   
to consider the Parikh image not only of a single regular language, but of the
intersection of multiple languages.
This happens in string solvers in
particular, as conjunctions of string constraints lead to the computation of
length images of intersections of regular languages represented as
finite-state automata.
On the other hand, we are often
not interested in a closed-form description of the complete Parikh image,
but rather in checking whether the Parikh image contains vectors satisfying
some given properties. 
\emph{The main problem considered in this paper is the following:}

{
    \centering
    \begin{tcolorbox}[colback=gray!5!white,colframe=gray!75!black,%
        title=Joint satisfiability of Parikh images modulo Presburger arithmetic constraints,%
        width=0.9\linewidth]
        Given regular languages~$\Language_1, \ldots, \Language_n$
        over a common alphabet $\Set{a_1, \ldots, a_k}$,
        and a Presburger arithmetic formula~$\phi(x_1, \ldots, x_k)$, 
        decide if there is a word~$w \in \Language_1 \cap
        \cdots \cap \Language_n$ whose Parikh vector 
        $\VectorLiteral{m_1, \ldots, m_k}$ satisfies $\phi$.
    \end{tcolorbox}
}

The problem can be solved using the classical construction of Parikh
images~\cite{muscholl-linear,generate-parikh-image} by first computing
the product of finite-state automata accepting the
languages~$\Language_1, \ldots, \Language_n$, then extracting a
Presburger arithmetic formula~$\rho$ describing the Parikh image of the product
automaton, and finally checking the satisfiability of $\phi \wedge \rho$.
  
While theoretically elegant, this construction has several
disadvantages that easily turn into bottlenecks when reasoning about
Parikh images in algorithms or applications.

Firstly, computing the product automaton is often prohibitively
expensive in both memory and CPU time.  In several instances we have
observed while solving real-world string constraints, the computation
of the product of automata exhausts the memory of any machine
available due to the exponential blow-up in size of the product,
quickly becoming intractable as the number of automata in the product
increases.

Secondly, the constructed Presburger arithmetic formula~$\rho$ contains a linear
number of existential quantifiers in the size of the product
automaton, as well as complex Boolean structure, which is needed to
express the connectedness of paths considered in the construction.
Solving formulas of this kind generally tends to be taxing for
solvers~\cite{ostrich-plus}; in the present case, since the product
automaton itself is exponentially big, also formula~$\rho$ has
exponential size in the number of considered regular languages, and
can easily become too complex for today's Presburger arithmetic
solvers to handle.
  
The current best published mitigation for this problem relies on the
observation that the Parikh image of an intersection of languages can
be over-approximated by the conjunction of the Parikh images of the
individual languages. Such over-approximation can be used to derive
that no word exists whose Parikh vector satisfies a formula~$\phi$,
falling back to computing the intersection of the languages (or of
some subset of the languages) for the satisfiable
cases~\cite{approximate-parikh}.

Addressing these concerns, we have developed a calculus for Parikh images of
intersections of regular languages that we call \Calculus{}. It allows us to
interleave the computation of product automata with the derivation of the
Parikh image. This enables the two
calculations to inform each other, eliminating unnecessary work, and pruning the
size of the partial products considered in the computation for a smaller memory
footprint.
Moreover, our approach allows us to do this precisely, without the expensive
fallbacks when approximations fail as in the case of~\cite{approximate-parikh}.
%
Our calculus is applicable not only to Parikh images,
but can be used for reasoning about the image of regular languages under
arbitrary homomorphisms into commutative monoids. Besides Parikh images,
this includes the length abstraction of regular languages commonly used in
string solvers.

Our primary influence for this work is the operation of (finite-domain) 
constraint programming (CP)%
~solvers~\cite{cp}. CP~solvers typically employ a combination of lazy enforcement
of constraints through \emph{propagation},
pruning values from decision variable domains that do not satisfy 
the constraints during solving, and \emph{branching}, splitting proofs
by cases in a way that will encourage maximum propagation.

For our problem, the constraints are automata connectivity and linear
equations (Presburger arithmetic formulas).
We use propagation to lazily enforce connectivity of automata,
removing transitions as they become unreachable during solving or are
eliminated by constraints in the Presburger formula. We
also branch on the presence or absence of transitions to
encourage propagation of the connectivity constraint. This allows
us to whittle down the automata
(including intermittent products) before moving on with potentially
 explosive product construction. The underlying observation
is that though the domain of our integer variables
is infinite, the number of simple paths
through the automata (and their product) is finite.
\emph{In summary, the key ideas are:}

{
    \centering
    \begin{tcolorbox}[colback=ourcolour!5!white,colframe=ourcolour!75!black,%
        title=Key paradigms in \Calculus{},%
        width=0.9\linewidth]

        \begin{itemize}
            \item Enforce automata connectivity constraints lazily;
            \item Use \emph{propagation} of constraints on the Parikh image to prune
            transitions from the automata that are incompatible with constraints;
            \item Intelligently \emph{branch} on the presence or absence of key
             transitions to drive propagation of the connectivity constraint;
             \item Compute products of automata lazily, after pruning transitions
             that violate constraints using propagation.
        \end{itemize}
    \end{tcolorbox}
  }

We implement \Calculus{} as a plug-in theory for the \Princess{} automated
theorem prover~\cite{princess}, and additionally wrap the Parikh image solver as a stand-alone tool, \Catra{}.
\Catra{} also supports a variant of the approximate method of~\cite{approximate-parikh}, its fall-back variant adapted
from~\cite{generate-parikh-image}, and an adapter for the \Nuxmv{} model
checker~\cite{nuxmv}. Using \Catra, we compare \Calculus{} to the other 
back-ends on \NrBenchmarks{} distinct Parikh automata intersection problems
generated by \OstrichPlus{} when solving the PyEx string constraint benchmark
suite involving string length constraints~\cite{pyex}, finding that \Calculus{}
outperforms both \Nuxmv{} and the baseline method, allowing \Calculus{}
to solve every problem solved by baseline within \SI{30}{s} in under \SI{5}{s}.
\Calculus{} in particular outperforms \Nuxmv{} on unsatisfiable instances, more than 
doubling the number of unsatisfiable results found within a \SI{30}{s}~timeout.

We also extend the \Ostrich{} string solver~\cite{ostrich} with a new back-end
based on \Catra{} as the Parikh image intersection solver of
the \OstrichPlus{}~algorithm, obtaining improvements in both
\textsc{Sat} and \textsc{Unsat} performance in \Ostrich{}' results
from SMT-COMP 2023, crucially allowing \Ostrich{} to win the
unsatisfiable category of the track of quantifier-free, linear
constraints over string logic (QF\_SLIA) and increasing the number of
\textsc{Sat}~results by \numprint{3}\% on the same track.

% PR: maybe only mention this in the evaluation?

% \Fudge{Others} have integrated it into the similarly
%\Princess{}-backed string constraint solver \OstrichPlus{}, where it is used to
%handle constraints that use cost-enriched automata, as described in
%\cite{ostrich-plus}.

In summary, we contribute:
\begin{itemize}
\item The \Calculus{} calculus to efficiently check the satisfiability
  of the Parikh image of an intersection of regular languages modulo
  Presburger arithmetic side conditions.
    \item Techniques to efficiently implement \Calculus{} in a modern
    automated theorem prover, including strategies for case splitting, clause
    learning, and constraint propagation for connectedness.
    \item The \Catra{} tool for solving such instances, containing an
    implementation of \Calculus{}, the over-approximation described
    in~\cite{approximate-parikh}, and an adapter for the~\Nuxmv{} model
    checker~\cite{nuxmv}.
    \item Experiments illustrating the performance of \Calculus{} on real-world
    examples from string solving, including \NrBenchmarks{} instances in a
    standardised format made available for future study.
\end{itemize}

\subsection{Related Work}

The problem of satisfying Parikh images over products of regular
languages, modulo Presburger arithmetic side conditions,
amounts to checking emptiness of products of
Parikh automata. Parikh automata are regular automata extended with integer
counters with given increments and decrements for each transition, where we
allow checking a set of linear constraints on the final values of the counters
(but not their intermittent values) \cite{parikh-automata}. Parikh automata
without constraints on the final values on their registers are also sometimes
called cost-enriched automata, weighted automata, or counter automata, depending
on exact definitions and side constraints. The decision problem tackled in this
paper, determining the emptiness of an intersection of Parikh automata, was
shown to be PSPACE-complete~\cite{graph-queries}.

Parikh image computations, as well as Parikh automata, feature extensively in
string solvers, including as mentioned above \Ostrich{} and \OstrichPlus{}
\cite{ostrich,ostrich-plus}, but also forms the basis of
\textsc{Trau}~\cite{trau-pldi},
and occurs in \textsc{Sloth}~\cite{sloth}. Parikh images frequently appear when
introducing cardinality constraints like length or string indexing.
While relying on Parikh images (or on being able to check
the emptiness of Parikh images under given side constraints), the mentioned
papers do not propose any techniques to compute Parikh images.

\citeauthor{muscholl-linear} and \citeauthor{generate-parikh-image}
define a closed-form description of the Parikh image of any regular
language as an existential Presburger arithmetic formula.
%
An approach to optimise the construction is to
over-approximate the Parikh image of a product of $k$~automata
$\ParikhMap(\Language(\Automaton_1) \cap \ldots \cap \Language(\Automaton_k))$
with the conjunction of the Parikh images, $\bigwedge_{i=1}^{k}
\ParikhMap(\Language(\Automaton_i))$~\cite{approximate-parikh}.
Due to the over-approximation,
this approach is primarily useful for unsatisfiable instances,
and requires falling back to computing the product of the automata before
using the standard approach for finding its image originally presented
in~\cite{generate-parikh-image}. Our calculus~\Calculus{}, in a somewhat
similar but more fine-grained manner, utilises laziness to postpone or
avoid the most expensive steps in the computation of Parikh images of
the intersection of regular languages.

Our calculus~\Calculus{} is also similar in spirit to the work
of \citeauthor{symbolic-boolean-derivatives}, who tackle the exponential
blow-up resulting from Boolean combinations
of finite-state automata in SMT string solvers
through the use of symbolic derivatives~\cite{symbolic-boolean-derivatives}.
 The research by
 \citeauthor{symbolic-boolean-derivatives} does not consider Parikh images,
 however. In addition to presenting a decision procedure for lazily dispatching
constraints, we similarly also allow for symbolic labelling of automata to
handle large alphabets.

Beyond the field of string solving, Parikh image computation is used
as an elementary building block in a variety of areas.  For instance,
Parikh automata have been proposed as the basis of queries in graph
databases~\cite{graph-queries}; Parikh images are used for handling
cardinalities in parameterised model checking for epistemic
logic~\cite{epistemic-logic}; or for handling summation constraints in
expressive array logics~\cite{rodrigoRaya}. Our calculus~\Calculus{},
and the stand-alone solver~\Catra{},
are potentially useful in all such applications.

Other generalisations of the Parikh image than the projections we use here
have been studied. Prominent examples include generalising the Parikh map to
segments of a fixed length~\cite{KARHUMAKI1980155} and the more general Parikh
matrix, which contains not only the Parikh vector, but also information about
the order of letters. Another notable generalisation is the p-vector, introduced
in~\cite{infinite-words}, which denotes the position of each letter in the word
rather than the number of their occurrences and allows for generalisations into
infinite alphabets. All of these in some sense extend the Parikh map. By
contrast, the main utility of the formulation introduced here is to reason
about Parikh images \emph{lazily,}
thereby potentially obtaining answers more quickly. We expect that our
calculus~\Calculus{} can be generalised to the mentioned functions on
formal languages as well, but leave such investigations to future work.


%%% Local Variables:
%%% mode: latex
%%% TeX-master: "main"
%%% End:
