A common problem in string constraint solvers is computing the Parikh image, a
linear arithmetic formula that describes all possible combinations of character
counts in strings of a given language. Automata-based string solvers frequently
need to compute the Parikh image of large products (intersections) of
nondeterministic automata, which in many operations is both prohibitively slow
and memory-intensive. We contribute a novel understanding of how Parikh maps can
be tackled as a constraint solving problem to solve real-world constraints
stemming from functions on regular languages, including length and indexing
constraints. Furthermore, we show how this formulation can be efficiently
implemented as a calculus, \Calculus{}, in an automated theorem prover
supporting Presburger logic.  We implement \Calculus{} in a tool called
\Catra{}, and evaluate it on constraints produced by the \OstrichPlus{} string
constraint solver when solving Parikh automata intersection problems produced
when solving standard string constraint benchmarks involving cardinality
constraints on strings. We show that our solution strictly outperforms the
standard approach described in~\citeauthor{generate-parikh-image} as well as the
over-approximating method recently described by~\citeauthor{approximate-parikh}
by a wide margin, and for realistic timeouts for constraint solving also
the~\Nuxmv{} model checker.