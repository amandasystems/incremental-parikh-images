A common problem in string constraint solvers is computing the Parikh image, a
linear arithmetic formula that describes all possible combinations of character
counts in strings of a given language. Automata-based string solvers frequently
need to compute the Parikh image of products (or intersections) of
finite-state automata, in particular when solving string constraints that
also include the integer data-type due to operations like string length
and indexing. In this context, the computation of Parikh images often turns
out to be both prohibitively slow
and memory-intensive. This paper contributes a new understanding of how
the reasoning about Parikh images can be cast as a constraint solving problem,
and questions about Parikh images be answered without explicitly computing
the product automaton or the exact Parikh image. The paper shows
how this formulation can be efficiently implemented as a calculus,
\Calculus{}, embedded in an automated theorem prover supporting
Presburger logic. The resulting standalone tool
\Catra{} is evaluate on constraints produced by the \OstrichPlus{} string
solver when solving standard string constraint benchmarks
involving integer operations.
The experiments show that \Calculus{} strictly outperforms the
standard approach by~\citeauthor{generate-parikh-image} to extract Parikh
images from finite-state automata, as well as the
over-approximating method recently described by~\citeauthor{approximate-parikh}
by a wide margin, and for realistic timeouts (under \SI{60}{s}) also
the~\Nuxmv{} model checker. When added as the Parikh image backend of \OstrichPlus{} to
the \Ostrich{} string constraint solver's portfolio, it boosts its results on the
quantifier-free strings with linear integer algebra track of SMT-COMP~2023 (QF\_SLIA)
enough to solve the most \Unsat{} instances in that track of all competitors.