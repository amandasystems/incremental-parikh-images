\subsection{Per-instance comparison between backends}
\begin{figure}[t]
  \centering
  \begin{subfigure}[b]{0.49\textwidth}
    \includegraphics[width=\textwidth]{graphs/\commit-duels-lazy-baseline-scatter.pdf}
    \caption{Per-instance runtime between baseline and \Calculus{}}
    \label{fig:duels:baseline}
    \Description[A scatter plot showing individual instances as dots compared to \Calculus{} and baseline]{The plot shows most dots at the bottom of the plot, with some plots collected at the right-hand side.}
  \end{subfigure}
  \hfill
  \begin{subfigure}[b]{0.49\textwidth}
    \includegraphics[width=\textwidth]{graphs/\commit-duels-lazy-nuxmv-scatter.pdf}
    \caption{Per-instance runtime between \Nuxmv{} and \Calculus{}}
    \label{fig:duels:nuxmv}
    \Description[A scatter plot showing individual instances as dots compared to \Calculus{} and \Nuxmv{}]{The plot shows flames of dots between the bottom and top, with a line of dots near the right-hand side corresponding to timeouts.}
  \end{subfigure}
\end{figure}

\subsection{Example instance for \Catra{}}
\lstinputlisting[caption={An example input file for \Catra{} for \cref{ex:string-constraints} in \cref{sec:intuition}, illustrating every major syntax element. From beginning to end: synchronised (product) automata using the keyword \texttt{synchronised} (automata A and B), labels (except those with ranges), register increments, and constraints on the final values of their counters.}, label=lst:input-example]{example.par}

%\subsection{SMT-LIB encoding of the string constraint example}
%\lstinputlisting[caption={An SMT-LIB encoding of \cref{ex:string-constraints}.}, label=lst:string-constraints]{example.smt2}
