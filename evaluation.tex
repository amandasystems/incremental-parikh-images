
We evaluate the performance of \Catra{} on~\NrBenchmarks{} instances of Parikh
automata intersection problems generated by the \OstrichPlus{} string constraint
solver~\cite{ostrich-plus} when solving the PyEx benchmarks, 
which are string constraints from symbolic execution that are known
to be hard for many solvers~\cite{pyex}. After generating an
initial~\InitialNrBenchmarks{}, we remove~\NrTrivial{} instances solved in under
five seconds by baseline as well as~\NrInvalid{} instances that were duplicates of other instances in the set.

 We also attempted to benchmark instances generated by
\OstrichPlus{} solving the Kaluza benchmarks~\cite{Saxena10:kaluza}
(\numprint{38227}~instances),
\iffalse
and pyex-len
(\numprint{791}~instances) benchmark suites\fi
but discarded them since they all turned out to be trivial for our tool
\Catra{}: every instance
could be solved in under five seconds, with a mean runtime of about 0.1s.
The benchmarks are run
on commit~\texttt{\commit} of \Catra{}\footnote{Github repository redacted for anonymity.}.

The benchmarks are executed in parallel across a cluster of identical
machines, \BenchmarkRig{}, as the sole job for each machine.  We compiled
the code using Scala~\ScalaVersion{} and executed the experiments
on~\JvmVersion{} with a maximum heap of~\MaxHeapSize{}. We used \Nuxmv{}
version~\NuxmvVersion{} invoked as a subprocess for each instance. Instances
were executed in batches of \BatchSize{}, each given a fresh JVM. Each JVM was
warmed up for \numprint{10}s on a random benchmark from the set before starting
to execute. We believe this represents a realistic use case where \Calculus{} is
used to support, e.g., a string solver. Experiments were executed in random order
for all backends. Each instance got a time budget of~\RuntimeTimeout.

All runtimes are measured in wall-clock time as observed by the JVM when
executing the instance, and exclude time spent parsing (usually far below
\numprint{0.1}s).

\subsection{Execution Time and Ability to Solve Instances}\label{sec:runtime}

In \cref{fig:solve-division}, we show how many of the~\NrBenchmarks{} instances
the respective back-end could solve. A summary of their outcomes by instance
type is also available in \cref{tab:solve-status}. Note that many instances lack
a ground truth as they are solved by only one backend. We see that
\Calculus{} generally outperforms \Nuxmv{} on determining unsatisfiability, as
does baseline, while being similar at satisfiable instances. Both \Calculus{} and
\Nuxmv{} outperforms baseline on satisfiable instances. On satisfiable
instances, \Nuxmv{} and \Calculus{} have similar performance.

Baseline performs worse on satisfiable instances because it executes a heuristic
meant to detect unsatisfiability early, similar to~\cite{approximate-parikh}.
The heuristic is enabled since the improvement is significant compared to the
extra cost.

\begin{table}
  \begin{center}
  \begin{tabular}{lrrr}
\toprule
backend & baseline & \Calculus & \Nuxmv \\
status after $30$s &  &  &  \\
\midrule
\textsc{Sat} & 6 & 3692 & 3288 \\
\textsc{Unsat} & 8958 & 15197 & 6356 \\
\bottomrule
\end{tabular}

  \end{center}
  \caption{Number of successful results within a timeout of \RuntimeTimeout{}.
  Instances solved by no backend within the timeout (about half of the set) are
  omitted from the table.}\label{tab:solve-status}
\end{table}

\begin{figure}[t]
  \centering
  \begin{subfigure}[b]{0.49\textwidth}
    \includegraphics[width=\textwidth]{graphs/\commit-by-solver.pdf}
    \caption{The division of statuses per backend.}
    \label{fig:solve-division}
    \Description[A bar chart showing three bars, one per backend, illustrating how many instances they could solve]{The bars are divided by satisfiable and unsatisfiable instances. Baseline could seemingly only solve unsatisfiable instances, PC* could solve a few satisfiable and mostly unsatisfiable, and nuxmv could solve about twice as many unsatisfiable as satisfiable instances, and slightly less in total than PC*.}
  \end{subfigure}
  \hfill
  \begin{subfigure}[b]{0.49\textwidth}
    \includegraphics[width=\textwidth]{graphs/\commit-time-boxplot.pdf}
    \caption{The distribution of runtimes for solved instances per backend. Total numbers of solved instances differ.}
    \label{fig:runtime-boxplot}
    \Description[A box plot showing the distribution of runtime over the three backends]{The middle box plot shows a tiny box centered around 0 seconds for the PC* backend, the rightmost box shows a bigger box between five seconds and 30 seconds for nuxmv, and the leftmost box shows a smaller box between 20 and 30 seconds for baseline. The whiskers of nuxmv span between 0 and the timeout, 60 seconds, while they are much tighter for PC*. On the other hand, PC* has a lot of outliers.}
  \end{subfigure}
  \vfill
  \begin{subfigure}[b]{\textwidth}
    \includegraphics[width=\textwidth]{graphs/\commit-cactus.pdf}
    \caption{The number of instances solved as the time budget increases, simulated from one \numprint{120}s-timeout run.}
    \label{fig:cactus}
    \Description[A cactus plot comparing the performance of nuxmv, PC* and baseline]{The line for baseline is strictly lower than the two others, and does not head upwards from zero until after 10 seconds of timeout, while the number of instances for nuxmv increases sharply until ca 10 seconds and then linearly after that. Lazy solves much more instances than nuxmv.}
  \end{subfigure}
  \caption{Evaluation of \Calculus{} solving Parikh automata problems generated by \Ostrich{}.}
\end{figure}

\subsubsection{Scalability}\label{sec:scaling}%
To investigate scaling we additionally execute \numprint{2000} randomly sampled
(without replacement) benchmarks with a \numprint{120}-second timeout. We add
runs of \Calculus{} with clause learning and restarts disabled, and with only
clause learning disabled to show the impact of \cref{sec:random-restarts}, and
\cref{sec:clause-learning} respectively.

A cactus plot showing the number of instances solved within a given timeout can
be seen in \cref{fig:cactus}. We see that \Calculus{} outperforms \Nuxmv{} in
general but that \Nuxmv{} might scale better on very long runtimes. This is
likely due to two factors. First, \Nuxmv{} is more mature than \Calculus{}, and
its more general model checking methods might pay off on more difficult
instances compared to problem-specific methods. Second, for longer-running
calculations clause learning as described in \cref{sec:clause-learning} might
matter more. As clause learning in \Catra{} is rudimentary and generalises
poorly across product computations, there is a performance hit.

\subsection{Evaluation in \Ostrich{}}\label{}%

\begin{table}
  \begin{center}
  \begin{tabular}{lrrrrrrr}
\toprule
solver & Competition & \Ostrich{} 1.3 & CA-Str & \Ostrich{}+CA & Z3-Noodler & cvc5 & z3alpha \\
kind &  &  &  &  &  &  &  \\
\midrule
SAT & 14217 & 9586 & 6427 & 9930 & 7344 & 14069 & 13599 \\
UNSAT & 7601 & 7356 & 5562 & 7387 & 5367 & 7377 & 7298 \\
\bottomrule
\end{tabular}

  \end{center}
  \caption{Number of solved benchmarks in the set of quantifier-free strings with
    linear integer arithmetic constraints (QF\_SLIA) at SMT-COMP~2023. The numbers are from
    the competition results, except for CA-Str, which is executed by us on the same 
    cluster as the competition with the same resources, and the two virtual portfolio solvers
    \Ostrich+CA and \textsc{Competition} which aggregate the best results from \Ostrich{}/CA-Str and 
    all the competition results for non-\Ostrich{} solvers respectively. }
  \label{tab:solve-status-smt-comp}
\end{table}

To evaluate the effectiveness of \Calculus{} in a string solver, \Catra{} was
experimentally integrated as the Parikh automata product solver into the string
solver \Ostrich{}~version~1.3. \Ostrich{} is an independently developed solver
that participated in the recent
SMT-COMP~2023\footnote{\url{https://smt-comp.github.io/2023/participants/ostrich}},
winning the single-query track for quantifier-free strings (\texttt{QF\_S}), as
well as dominating other solvers on unsatisfiable string
benchmarks~\cite{smt-comp-23}.

 At SMT-COMP, \Ostrich{} competed using a portfolio of three different
 back-ends:
\begin{itemize}
\item \textbf{BW-Str:} a backward-propagation-based solver, not utilising
  Parikh automata~\cite{ostrich}.
\item \textbf{ADT-Str:} a solver based on algebraic data-types.
\item \textbf{CE-Str:} a re-implementation of the \OstrichPlus{}
  algorithm~\cite{ostrich-plus}, using Parikh automata and the
  encoding from~\cite{generate-parikh-image} (referred to in this paper as baseline).
\end{itemize}

For our experiments, we modified the CE-Str solver to apply \Catra{}
 (with \Calculus{} as its chosen backend)
instead of the previous baseline method, resulting in a new back-end
\textbf{CA-Str}. Combining the results from SMT-COMP for \Ostrich{}~1.3 
with our new results running CA-Str we constructed a virtual 
portfolio~\Ostrich{}+CA that simulated running CA-Str as a fourth
back-end to \Ostrich{}.

%As benchmarks, we again focused on the PyEx benchmarks~\cite{pyex}, of
%which we picked 1000~problems uniformly at random. We ran
%different combinations of the
%back-ends on those 1000~problems with a 120s timeout.
%
%The \texttt{QF\_S} track contains no Parikh automata intersection problems,
%however. For that, we need to look to the linear integer arithmetic
%(\texttt{QF\_SLIA}), where \Ostrich{} did worse in SMT-COMP. The evaluations in
%\cref{sec:scaling} use inputs generated by an older branch of CEFA-enriched
%\Ostrich{} on the PyEx benchmarks, which are part of the set used in the
%competition. In this section, we evaluated a version of \Ostrich{} using
%\Catra{} as its back-end \Fudge{on StarExec, the same cluster SMT-COMP ran on}
%on \numprint{1000}~randomly selected instances from the PyEx benchmarks. 
%
% comparison \Fudge{to the other provers of the 2023
%competition on the same benchmarks, as well as \Ostrich{} using the baseline
%back-end described in \cref{sec:implementing-baseline}}.

We extend the results of SMT-COMP 2023 with our modified \Ostrich{}
on the single-query string solving with linear integer arithmetic constraint
track (QF\_SLIA) in~\cref{tab:solve-status-smt-comp}. We obtain
the results by executing \Ostrich{}~1.3 with CA-Str using
the same benchmarking infrastructure (the StarExec cluster)
and configuration that ran SMT-COMP 2023, combining our new results for CA-Str with the
published results of SMT-COMP 2023. Where avaiable, we use the revised, out-of
competition version of the results for solvers with bugs, including \Ostrich{} and Z3-Noodler.
Note that Z3-Noodler abstained on 70~instances, and thus has a lower total number of results.

We picked QF\_SLIA since
\Ostrich{} already performed well on the other two string solving tracks. In fact,
every solver, including \Ostrich{}, handled every benchmark in QF\_SNIA within the timeout.
%
As can be seen from \cref{tab:solve-status-smt-comp}, integrating \Catra{} as a back-end
leads to gains both on satisfiable and unsatisfiable problems. On
satisfiable problems, the combination \Ostrich+CA is still
outperformed by cvc5 and z3alpha. On unsatisfiable benchmarks,
\Ostrich+CA narrowly beats the other solvers,
which is promising given that \Ostrich{}~1.3 (without CA-Str), cvc5, and z3alpha show very strong
performance on this class of benchmarks. Similar improvements
occur in QF\_S, they are much smaller (an additional 2 unsatisfiable, 4 satisfiable 
solved out of~8\,847 in total).

The results show that \Calculus{} can be used to enhance the performance of an
automaton-based string solver. The results should be considered preliminary,
however, as we believe that a deeper integration of \Catra{} into string solvers
can lead to significant performance gains. In particular, the integration layer 
is currently too shallow to allow \Ostrich{} to learn clauses generated by \Catra{}
and additionally incurs subprocess-spawning overhead and others, as well as overhead
from serialising and deserialising the current Parikh automata problem into \Catra{}'s
input format.

\subsection{Threats to Validity}

The most obvious threat to validity would be an unsound implementation. To
address this we have validated all reported solutions made by \Calculus{} with
\Nuxmv{}. A previous version contained a race with random restarts during
product materialisation causing non-deterministic unsoundness in \numprint{0.7}
of instances. Since addressing that bug we have observed no further soundness
issues, with the exception of one erroneous \textsc{Unsat} response on the QF\_S track of 
SMT-COMP (\texttt{QF\_S\/20230329-automatark-lu\/instance08425.smt2}) that likely occurred
in the interaction surface between \Ostrich{} and \Catra{}, since we have robustly tested 
\Catra{} separately from \Ostrich{}. For this case, as well as for the similarly buggy case
for Z3-Noodler, we use the same rule as SMT-COMP and count the erroneous responses as error
results.

The second threat to validity is our implementation of automata operations. As
\Calculus{} by design offloads some of the product computation work onto
\Princess{}, the baseline could be unfairly disadvantaged by a slow automata
product implementation. We believe this is not an issue since similar
performance issues with the baseline approach have been reported for string
solvers. Additionally, profiling shows that baseline spends most of its time in
\Princess{}, suggesting that the automaton implementation is not the bottleneck.
Finally, the difference in performance between \Calculus{} and baseline has been
robust under significant optimisation of the automata library.

%%% Local Variables:
%%% mode: latex
%%% TeX-master: "main"
%%% End:
