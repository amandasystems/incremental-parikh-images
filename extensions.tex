
This section describes a number of possible extensions to the calculus. Note
that some are partially or fully implemented in \Catra{} already. In particular,
we have some level of support for symbolic transitions over Unicode alphabets to
keep practical automata under a reasonable size, though we do not allow a full Boolean algebra over symbols as described e.g. in~\cite{symbolic-automata}.

\subsection{Backjumping and Learning No-Goods}\label{sec:ext:backjumping}

\Calculus{} can be accelerated for some instances by adding rules for
backjumping. In particular, key connectedness constraints for an automaton can
be learnt. In that case, when discovering that a transition $\Transition$ of an
automaton $\Automaton$ under a certain transition variable $\Filter$ has become
unreachable, we can learn the clause $\Connected(\Automaton, \Filter) \land
\bigwedge_{\Transition' \in C} \Filter(\Transition') = 0 \implies \Transition = 0$ where
$\Dominates(C, \Automaton, \Transition)$ for some minimal $C$. Note that this
captures both forward (unreachable from the initial state) and
backward (does not reach an accepting state) unreachability. Bidirectional cut
learning is implemented in \Catra, and has provided a slight improvement in
performance on some instances.

The other source of clauses to learn is the \Materialise{} rule. Whenever an
attempt to materialise a product of two ($\Filter$-filtered) automata
produces an empty product, we can determine
the cause of the failure with respect to the automata and their respective
transition variables, and learn no-good combinations that can never be part of a
model. In order to be able to do this, we need a few semantic predicates to
record the state of the calculation, as well as a system of disambiguating
automata, since it is possible to arrive at the same automaton by multiple
combinations of decisions and products. In \Catra{}, we use a number of
additional predicates to record the status of the materialisation of products,
to separate instances of our main predicate, and to register the mapping between
automata and their respective transition $\Filter$ terms, described in
\cref{sec:implementation}. However, at this point only rudimentary no-good
learning is implemented.

\subsection{Symbolic Automata}\label{sec:ext:symbolic}

Extending \Calculus{} to support fully symbolic automata is possible within the
framework, depending on your choice of $\Map$. The difficulty consists in
handling the mapping of the homomorphism $\Map$ over symbolic labels, assuming
it maps to finitely many monoid elements. This is not always straightforward, as
seen in the example of \cref{sec:multiple:example}. This complexity is inherent
in computing the full Parikh image, and stems from the fact that we need to
differentiate between the possible interpretations of the $\Sigma$ transitions
without knowing ahead of times which ones will actually be materialised in the
product. If, on the other hand, if our $\Map$ had been length-counting, which
does not differentiate between values, it would have required no adaptation at
all. With a somewhat liberal interpretation of what we are allowed to map to
(e.g. fresh terms), it is possible to use $\Map$ to represent choice operations
like the ones in a range label. In \Catra{}, we frontload this problem by
requiring the user to encode their input as a Parikh automaton. For an example of how automata can be encoded, see \cref{sec:parikh-automata}.

\subsection{Transducers}\label{sec:ext:transducers}

Another interesting application is applying \Calculus{} to automata with
multiple tracks, e.g.\ transducers. One application of such a calculus would be
to represent replace operations and other functions on regular languages, and to
be able to answer questions such as \enquote{does this operation change the length of the
string}. The difficulty in implementing it for transducers is first to perform
the mapping on the labels, which for both the length and Parikh cases is
straightforward; just do the same thing in two dimensions. The more complicated
operation is defining the product of transducers. In some cases it would
probably be desirable to perform synchronisation (e.g.\ requiring overlapping
transitions) on only some transitions, for example the first track.

Implementing such a calculus is straightforward in \Calculus, since the
definition of products was left out of the definition. All that would be needed
is an appropriate update of the definitions of products. Similarly, \Catra was
written with modularity in mind, and it should be straightforward to extend both
the input grammar and automata implementations to accommodate multi-track
automata with arbitrary synchronisation.

\subsection{Finding the Presburger representation of a homomorphic image}\label{sec:finding-the-image}

To find the Presburger form of the homomorphic image efficiently, we adapt the
quantifier elimination approach of~\cite{qe} to our problem domain. The core
method is the same: we incrementally use \Calculus{} to find models,
\Generalise{} the models we find into a quantifier-free Presburger formula with
the $\MonoidElement$ as the only free variable (algorithm~\ref{alg:generalise}), add the
negated formula as a constraint, and continue enumerating models until none
remain. The disjunction of the generalised models we enumerated is now our
image.

\begin{algorithm}
  \caption{$\Generalise{}(\Automaton, \Filter, a)$ will generalise a final product $\Automaton$ and a model $a$ under the homomorphism $\Map$.}\label{alg:generalise}
  \KwData{$\Automaton$, a product from \Calculus{}, a model $a$ assigning counts to the terms that $\Filter$ associate with each transition of $\Automaton$, and our homomorphism $\Map$ that we want to compute the image modulo.}
  \KwResult{a quantifier-free Presburger formula $P$ representing a partial $\Map$-homomorphic image}
  \SetKwFunction{EliminateQuantifiers}{eliminateQuantifiers}

  $\Automaton' \gets \Tuple{\Automaton_\States, \Automaton_{\InitialState}, \Automaton_\AcceptingStates, \Set{\Transition \in \Automaton_\Transitions \SuchThat a(\Filter(\Transition)) \neq 0}}$

  \KwRet{\EliminateQuantifiers{$\Map(\ParikhMap(\Automaton'))$}}
  \end{algorithm}


  \begin{algorithm}
    \DontPrintSemicolon
    \caption{$\FindImage{}(\Automaton_1 \times \ldots \times \Automaton_k, \Map)$ will find the Presburger form for the product $\Automaton_1 \times \ldots \times \Automaton_k$ modulo a homomorphism $\Map$ where the only free variable is/are the one(s) representing the monoid element of $\Map$.}\label{alg:find-image}
    \KwData{$\Automaton$, a product from \Calculus{}, a model $a$ assigning counts to the terms that $\Filter$ associate with each transition of $\Automaton$, and our homomorphism $\Map$ that we want to compute the image modulo.}
    \KwResult{a quantifier-free Presburger formula $P$ representing a partial $\Map$-homomorphic image}
    \SetKwFunction{NewTheoremProver}{newTheoremProver}
    \SetKwFunction{EliminateQuantifiers}{eliminateQuantifiers}
    \SetKwFunction{FreshVariable}{freshVariable}
    \SetKwFunction{Assert}{assert}
    \SetKwFunction{GetModel}{getModel}
    \SetKwData{ImageVar}{image}
  
$p \gets \NewTheoremProver{}$\;
$\Filter(\Transition) := \FreshVariable{p}$ for every $\Transition \in \Transitions_{\Automaton}$\;
$\MonoidElement \gets$ \FreshVariable{$p$}\;
\Assert{$p, \exists \MonoidElement, \Filter(\Transition) \text{ for every } \Transition \in \Transitions_\Automaton \HoldsThat \ImagePredicate{\Automaton}{\Map}{\Filter}{\MonoidElement} \land \AndComp{\Transition \in \Transitions}{\Filter(\Transition) \geq 0}$}\;
$\ImageVar \gets \bot$\;
\While{$p$ has more models}{
  $\Tuple{\Automaton, a} \gets \GetModel(p)$\;
  $G \gets \Generalise{}(\Automaton, \Filter, a)$\;
  $\ImageVar \gets G \lor \Image$\;
  \Assert{$p, \lnot G$}\;
  }
    \KwRet{\ImageVar}
    \end{algorithm}

    The \GetModel{} function is nonstandard in that it returns both the model
    and its associated automaton. Since algorithm~\ref{alg:find-image} is
    essentially a form of quantifier elimination and therefore an internal affair
    to the theorem prover, this should be considered fair game.
    
    A preliminary implementation of this approach is available as part
    of~\Catra, but has been neither optimised nor fully tested.


%%% Local Variables:
%%% mode: latex
%%% TeX-master: "main"
%%% End:
